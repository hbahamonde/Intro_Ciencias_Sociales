%This is a LaTeX template for homework assignments
\documentclass{article}
\usepackage{amsmath}
\usepackage{import} % To import email.

\usepackage{hyperref}
\hypersetup{
    bookmarks=true,         % show bookmarks bar?
    unicode=false,          % non-Latin characters in Acrobat’s bookmarks
    pdftoolbar=true,        % show Acrobat’s toolbar?
    pdfmenubar=true,        % show Acrobat’s menu?
    pdffitwindow=true,     % window fit to page when opened
    pdfstartview={FitH},    % fits the width of the page to the window
    pdftitle={My title},    % title
    pdfauthor={Author},     % author
    pdfsubject={Subject},   % subject of the document
    pdfcreator={Creator},   % creator of the document
    pdfproducer={Producer}, % producer of the document
    pdfkeywords={keyword1} {key2} {key3}, % list of keywords
    pdfnewwindow=true,      % links in new window
    colorlinks=true,       % false: boxed links; true: colored links
    linkcolor=blue,          % color of internal links (change box color with linkbordercolor)
    citecolor=blue,        % color of links to bibliography
    filecolor=blue,      % color of file links
    urlcolor=blue           % color of external links
}


\usepackage{multido}
%\newcommand{\cmd}{-x-}
\newcommand{\Repeat}{\multido{\i=1+1}}


\usepackage{geometry}
\geometry{
  %body={6.5in, 8.5in},
  left=0.7in,
  right=0.7in,
  top=0.7in,
  bottom=1in
}


\begin{document}
\subsection*{Gira esta p\'agina s\'olo cuando el profesor/ayudante lo indique.}

\clearpage
\newpage

{\centering\section*{Ensayo Secci\'on Sociedad\\(AP)(CO)1000-1 ``Introducci\'on a las Ciencias Sociales''}}

{\vspace{.5cm}\raggedright{\bf Nombre}: \line(1,0){200}}. %you can change the length of the lines by changing the number in the curly brackets
{\vspace{.5cm}\hspace{4.5cm}\raggedright{\bf Fecha}: \line(1,0){100}}. %you can change the length of the lines by changing the number in the curly brackets


{\vspace{.5cm}\raggedright \bf Profesor}: H\'ector Bahamonde, PhD.\\
%{\bf Clase}: {\input{/Users/hectorbahamonde/RU/Teaching/Intro_Ciencias_Sociales/time_class_1.txt}\unskip}; {\input{/Users/hectorbahamonde/RU/Teaching/Intro_Ciencias_Sociales/time_class_2.txt}\unskip}.\\
%{\bf Sala}: C306.


\vspace{0.5cm}\subsection*{Instrucciones Generales}

El examen se realizar\'a el \underline{{\input{/Users/hectorbahamonde/RU/Teaching/Intro_Ciencias_sociales/date_seccion_sociedad.txt}\unskip}}, en horario de clases. Ex\'amenes que se entreguen despu\'es de esta hora/d\'ia, no ser\'an recibidos, y tendr\'an una nota 1. El examen tiene {\input{/Users/hectorbahamonde/RU/Teaching/Intro_Ciencias_sociales/num_ensayos_seccion.txt}\unskip} preguntas tipo ensayo. El examen vale {\input{/Users/hectorbahamonde/RU/Teaching/Intro_Ciencias_sociales/percentage_prueba_seccion.txt}\unskip}\% de la nota final, y toda la materia de la secci\'on podr\'ia ser preguntada. Las preguntas han sido sacadas de las lecturas, documentales, discusiones en clase y ayudant\'ia. No se permiten materiales de apoyo de ning\'un tipo. Bajo ninguna circunstancia se permitir\'an apuntes, libros, o pedazos de papel escritos, etc. El uso de aparatos electr\'onicos (laptops, relojes inteligentes, celulares, tablets, etc.) est\'a estrictamente prohibido. Si necesitas usar el servicio higi\'enico, tendr\'as que dejar tu celular con el profesor. Cualquier violaci\'on a estas reglas ser\'a sancionada con un 1 en la prueba.  Escritura dif\'icil de leer quitar\'a puntos. Usa oraciones enteras: no se permite el uso de punt\'eos, o esquemas. Devuelve todas las hojas al final de la prueba. Puedes salir de la sala en cualquier minuto, pero despu\'es no ser\'a posible entrar a la sala nuevamente. {\bf La copia no ser\'a tolerado}: si existe evidencia de copia, tendr\'as un 1 en toda la prueba ({\input{/Users/hectorbahamonde/RU/Teaching/Intro_Ciencias_Sociales/percentage_prueba_seccion.txt}\unskip}\% de la nota final), sin derecho a apelaci\'on. {\bf Deber\'as dejar \emph{todo} debajo del pizarr\'on (mochila, estuche, celular, etc.). S\'olo podr\'as tener contigo: agua (si es que quieres), y un l\'apiz.}


 

\subsection*{Ensayos}

El ensayo debe tener {\bf al menos 1.5}, pero {\bf nunca m\'as de 2 p\'aginas}. Respuestas m\'as cortas que el l\'imite m\'inimo, obtendr\'an un 1 en ese ensayo. El m\'inimo oficial est\'a denotado por un s\'imbolo ``$\star$''. Por favor, responde la pregunta. Esto es, (1) no reescribas, replantees, ni cambies la pregunta en tu ensayo, (2) evita dar respuestas indirectas. Cualquier informaci\'on que est\'e m\'as all\'a del m\'aximo, no ser\'a le\'ida. En total, habr\'a {\input{/Users/hectorbahamonde/RU/Teaching/Intro_Ciencias_Sociales/num_ensayos_seccion.txt}\unskip} ensayo. Si hay m\'as de {\input{/Users/hectorbahamonde/RU/Teaching/Intro_Ciencias_Sociales/num_ensayos_seccion.txt}\unskip} ensayo, s\'olo se leer\'an el primero.% {\input{/Users/hectorbahamonde/RU/Teaching/Intro_Ciencias_Sociales/num_ensayos_seccion.txt}\unskip}. %Las respuestas no necesariamente deben estar en orden correlativo.


\line(1,0){510}

{\bf Por favor \underline{responde} la siguiente pregunta:} 


\begin{enumerate}
    \item {\input{/Users/hectorbahamonde/RU/Teaching/Teaching_Questions_Database/Intro_Ciencia_Social/8.txt}\unskip}
    
\end{enumerate}    

\clearpage
\newpage

% First Essay
\subsection*{Ensayo N\'umero \line(1,0){40}.}
\Repeat{24}{\line(1,0){510}\vspace{0.5cm}\\}
\clearpage
\newpage
\Repeat{12}{\hspace{-5mm}\line(1,0){510}\vspace{0.5cm}\\}
{$\star$}\\
\Repeat{12}{\hspace{-5mm}\line(1,0){510}\vspace{0.5cm}\\}
%\Repeat{24}{\line(1,0){510}\vspace{0.5cm}\\}
\clearpage
\newpage


% Organization 1
\subsection*{Esta secci\'on no sera le\'ida ni corregida. Esta secci\'on es para que tu organices tus respuestas (dibujando esquemas, por ej.), si fuera necesario.}
\clearpage
\newpage

\end{document}