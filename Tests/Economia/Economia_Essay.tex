%This is a LaTeX template for homework assignments
\documentclass{article}
\usepackage{amsmath}
\usepackage{import} % To import email.

\usepackage{hyperref}
\hypersetup{
    bookmarks=true,         % show bookmarks bar?
    unicode=false,          % non-Latin characters in Acrobat’s bookmarks
    pdftoolbar=true,        % show Acrobat’s toolbar?
    pdfmenubar=true,        % show Acrobat’s menu?
    pdffitwindow=true,     % window fit to page when opened
    pdfstartview={FitH},    % fits the width of the page to the window
    pdftitle={My title},    % title
    pdfauthor={Author},     % author
    pdfsubject={Subject},   % subject of the document
    pdfcreator={Creator},   % creator of the document
    pdfproducer={Producer}, % producer of the document
    pdfkeywords={keyword1} {key2} {key3}, % list of keywords
    pdfnewwindow=true,      % links in new window
    colorlinks=true,       % false: boxed links; true: colored links
    linkcolor=blue,          % color of internal links (change box color with linkbordercolor)
    citecolor=blue,        % color of links to bibliography
    filecolor=blue,      % color of file links
    urlcolor=blue           % color of external links
}


\usepackage{multido}
%\newcommand{\cmd}{-x-}
\newcommand{\Repeat}{\multido{\i=1+1}}


\usepackage{geometry}
\geometry{
  %body={6.5in, 8.5in},
  left=0.7in,
  right=0.7in,
  top=0.7in,
  bottom=1in
}


\begin{document}
%\subsection*{Gira esta p\'agina s\'olo cuando el profesor lo indique.}
%
%\clearpage
%\newpage

{\centering\section*{Ensayo Secci\'on Econom\'ia\\(AP)(CO)1000-1 ``Introducci\'on a las Ciencias Sociales''}}

{\vspace{.5cm}\raggedright{\bf Nombre}: \line(1,0){200}}. %you can change the length of the lines by changing the number in the curly brackets
{\vspace{.5cm}\hspace{4.5cm}\raggedright{\bf Fecha}: \line(1,0){100}}. %you can change the length of the lines by changing the number in the curly brackets


{\vspace{.5cm}\raggedright \bf Profesor}: H\'ector Bahamonde, PhD.\\
%{\bf Clase}: {\input{/Users/hectorbahamonde/RU/Teaching/Intro_Ciencias_Sociales/time_class_1.txt}\unskip}; {\input{/Users/hectorbahamonde/RU/Teaching/Intro_Ciencias_Sociales/time_class_2.txt}\unskip}.\\
%{\bf Sala}: C306.


\vspace{0.5cm}\subsection*{Instrucciones Generales}


El temario del ensayo se entregar\'a, a m\'as tardar, el 17 de mayo a las 9 am v\'ia \texttt{uCampus}. Deber\'as entregar tus respuestas en formato \emph{MS Word} el \underline{{\input{/Users/hectorbahamonde/RU/Teaching/Intro_Ciencias_Sociales/date_seccion_economia.txt}\unskip}} a las 23.00 horas v\'ia \texttt{uCampus} tambi\'en. {\bf Bajo ninguna circunstancia se recibir\'an ensayos impresos, ni entregados v\'ia e-mail}. No se procesar\'an preguntas durante el tiempo que est\'es escribiendo el ensayo. Los ensayos que se entreguen despu\'es de esta hora/fecha, no ser\'an recibidos, y tendr\'an una nota 1, autom\'aticamente. Si tienes problemas de conectividad (i.e. tu Internet es lento), planifica tu trabajo para que alcances a cumplir con el plazo. Si no tienes acceso a un computador, deber\'as encontrar soluciones alternativas. Una posible soluci\'on ser\'ia trabajar en el laboratorio de computaci\'on UOH. Si entre medio existen feriados y/o fines de semana, planifica tu trabajo para que alcances a cumplir con el plazo. {\bf Siempre recuerda que el plazo es final}. El ensayo tiene {\input{/Users/hectorbahamonde/RU/Teaching/Intro_Ciencias_Sociales/num_ensayos_seccion.txt}\unskip} preguntas tipo ensayo. El ensayo vale {\input{/Users/hectorbahamonde/RU/Teaching/Intro_Ciencias_Sociales/percentage_prueba_seccion.txt}\unskip}\% de la nota final, y toda la materia de la secci\'on podr\'ia ser preguntada. Las preguntas han sido sacadas de las lecturas, documentales, discusiones en clase y ayudant\'ia. {\bf No se permiten materiales de apoyo de ning\'un tipo: s\'olo usa el material bibliogr\'afico del curso}. Bajo ninguna circunstancia uses referencias externas (Internet, otras fuentes, etc.). Usa oraciones enteras: no se permite el uso de punt\'eos, o esquemas. 


 

\subsection*{Ensayos}

Deber\'as escribir {\input{/Users/hectorbahamonde/RU/Teaching/Intro_Ciencias_Sociales/num_ensayos_seccion.txt}\unskip} ensayos en total. {\bf Por cada uno}, deber\'as escribir {\bf al menos 2.000} palabras (sin espacios), pero {\bf nunca m\'as de 2.500 palabras} (sin espacios), y en \underline{tama\~no de letra 12}. Es decir, el largo \emph{total} de tu trabajo (los {\input{/Users/hectorbahamonde/RU/Teaching/Intro_Ciencias_Sociales/num_ensayos_seccion.txt}\unskip} ensayos) tendr\'a entre 6.000 y 7.500 palabras. Para contar cu\'antas palabras llevas, ocupa la funci\'on cargada en \emph{MS Word}. Puedes ver un tutorial online \href{http://www.youtube.com/watch?v=I-dHsTvLe3M}{aqu\'i}. Respuestas m\'as cortas que el l\'imite m\'inimo, obtendr\'an un 1 en ese ensayo. Por favor, responde la pregunta. Esto es, (1) no reescribas, replantees, ni cambies la pregunta en tu ensayo, (2) evita dar respuestas indirectas. Cualquier informaci\'on que est\'e m\'as all\'a del m\'aximo, no ser\'a le\'ida. Si hay m\'as de {\input{/Users/hectorbahamonde/RU/Teaching/Intro_Ciencias_Sociales/num_ensayos_seccion.txt}\unskip} ensayos, s\'olo se leer\'an los primeros {\input{/Users/hectorbahamonde/RU/Teaching/Intro_Ciencias_Sociales/num_ensayos_seccion.txt}\unskip}. Tus respuestas no necesariamente deben estar en orden correlativo.
\\
\\
Presta atenci\'on a las siguientes instrucciones:

\begin{itemize}
	\item {\bf Citas}: no son necesarias. Pero si decides citar, independiente del formato que ocupes (ISI, APA, etc.), h\'azlo a pi\'e de p\'agina. {\bf \underline{Recuerda}: toda informaci\'on que no haya sido redacta por ti, \underline{debe} ser citada}.
	\item {\bf El plagio no ser\'a tolerado}: si existe evidencia de plagio, tendr\'as un 1 en el ensayo completo de la secci\'on de Econom\'ia ({\input{/Users/hectorbahamonde/RU/Teaching/Intro_Ciencias_Sociales/percentage_prueba_seccion.txt}\unskip}\% de la nota final), sin derecho a apelaci\'on.
	\item {\bf Bibliograf\'ia}: no es necesaria. Pero si decides citar, es natural que pongas la lista de textos referenciados (i.e. una bibliograf\'ia) al final de tu trabajo. Los l\'imites de palabras estipulados no consideran la secci\'on de Bibliograf\'ia.
	\item {\bf Portada}: no es necesaria. Pero si decides tener una, los l\'imites de palabras estipulados no consideran la secci\'on de Portada. {\bf No olvides poner tu nombre y carrera}.
	\item {\bf Formato}: s\'olo se aceptar\'an archivos de \emph{MS Word} (i.e. con extensi\'on \texttt{.doc} o \texttt{.docx}). Tama\~no de letra es 12. Escoge un tipo de letra y m\'argenes razonables.
\end{itemize}

\line(1,0){510}

\clearpage
\newpage

{\bf Por favor \underline{responde} las siguientes DOS preguntas:} 


\begin{enumerate}
    \item {\input{/Users/hectorbahamonde/RU/Teaching/Teaching_Questions_Database/Intro_Ciencia_Social/1.txt}\unskip}
    
    \item {\input{/Users/hectorbahamonde/RU/Teaching/Teaching_Questions_Database/Intro_Ciencia_Social/2.txt}\unskip}
\end{enumerate}    

\line(1,0){510}


{\bf Por favor \underline{escoge} UNA de las siguientes dos preguntas:} 


\begin{enumerate}
\setcounter{enumi}{2}
    \item {\input{/Users/hectorbahamonde/RU/Teaching/Teaching_Questions_Database/Intro_Ciencia_Social/3.txt}\unskip} 
    
   \item {\input{/Users/hectorbahamonde/RU/Teaching/Teaching_Questions_Database/Intro_Ciencia_Social/4.txt}\unskip} 
\end{enumerate}    

\line(1,0){510}


\clearpage

\end{document}