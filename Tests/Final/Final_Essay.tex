%This is a LaTeX template for homework assignments
\documentclass{article}
\usepackage{amsmath}
\usepackage{import} % To import email.

\usepackage{hyperref}
\hypersetup{
    bookmarks=true,         % show bookmarks bar?
    unicode=false,          % non-Latin characters in Acrobat’s bookmarks
    pdftoolbar=true,        % show Acrobat’s toolbar?
    pdfmenubar=true,        % show Acrobat’s menu?
    pdffitwindow=true,     % window fit to page when opened
    pdfstartview={FitH},    % fits the width of the page to the window
    pdftitle={My title},    % title
    pdfauthor={Author},     % author
    pdfsubject={Subject},   % subject of the document
    pdfcreator={Creator},   % creator of the document
    pdfproducer={Producer}, % producer of the document
    pdfkeywords={keyword1} {key2} {key3}, % list of keywords
    pdfnewwindow=true,      % links in new window
    colorlinks=true,       % false: boxed links; true: colored links
    linkcolor=blue,          % color of internal links (change box color with linkbordercolor)
    citecolor=blue,        % color of links to bibliography
    filecolor=blue,      % color of file links
    urlcolor=blue           % color of external links
}


\usepackage{multido}
%\newcommand{\cmd}{-x-}
\newcommand{\Repeat}{\multido{\i=1+1}}


\usepackage{geometry}
\geometry{
  %body={6.5in, 8.5in},
  left=0.7in,
  right=0.7in,
  top=0.7in,
  bottom=1in
}


\begin{document}
%\subsection*{Gira esta p\'agina s\'olo cuando el profesor/ayudante lo indique.}
%
%\clearpage
%\newpage

{\centering\section*{Ensayo Final\\(AP)(CO)1000-1 ``Introducci\'on a las Ciencias Sociales''}}

{\vspace{.5cm}\raggedright{\bf Nombre}: \line(1,0){200}}. %you can change the length of the lines by changing the number in the curly brackets
{\vspace{.5cm}\hspace{4.5cm}\raggedright{\bf Fecha}: \line(1,0){100}}. %you can change the length of the lines by changing the number in the curly brackets


{\vspace{.5cm}\raggedright \bf Profesor}: H\'ector Bahamonde, PhD.\\
%{\bf Clase}: {\input{/Users/hectorbahamonde/RU/Teaching/Intro_Ciencias_Sociales/time_class_1.txt}\unskip}; {\input{/Users/hectorbahamonde/RU/Teaching/Intro_Ciencias_Sociales/time_class_2.txt}\unskip}.\\
%{\bf Sala}: C306.


\vspace{0.5cm}\subsection*{Instrucciones Generales}

Deber\'as entregar el ensayo v\'ia uCampus seg\'un lo establecido en el programa de curso. Ex\'amenes que se entreguen despu\'es de esta hora/d\'ia, no ser\'an recibidos, y tendr\'an una nota 1. Entra toda la materia del semestre (incluyendo la secci\'on Introducci\'on). {\bf El plagio no ser\'a tolerado}: si existe evidencia de plagio, tendr\'as un 1, sin derecho a apelaci\'on. Ocupa un sistema libre de citar (APA, ISI, MLE, etc.). Cade vez que cites, no lo hagas por m\'as de dos l\'ineas. Ocupa un m\'aximo de 5 citas.


 

\subsection*{Ensayo}

El ensayo debe tener {\bf al menos 1.5}, pero {\bf nunca m\'as de 2 p\'aginas} tama\~no carta, letra tama\~no 12, letra Arial. Respuestas m\'as cortas que el l\'imite m\'inimo, obtendr\'an un 1 en ese ensayo. Por favor, responde la pregunta. Esto es, (1) no reescribas, replantees, ni cambies la pregunta en tu ensayo, (2) evita dar respuestas indirectas. Cualquier informaci\'on que est\'e m\'as all\'a del m\'aximo, no ser\'a le\'ida.


\line(1,0){510}

{\bf Por favor \underline{responde} la siguiente pregunta:} 


\begin{enumerate}
    \item Cu\'al es el rol de las ciencias sociales en el cambio social? Aseg\'urate de integrar en tu respuesta aspectos relacionados a los or\'igenes de la desigualdad, el desarrollo economico, politico y social de las naciones. 
    
\end{enumerate}    

\clearpage
\newpage

% First Essay
%\subsection*{Ensayo N\'umero \line(1,0){40}.}
%\Repeat{24}{\line(1,0){510}\vspace{0.5cm}\\}
%\clearpage
%\newpage
%\Repeat{12}{\hspace{-5mm}\line(1,0){510}\vspace{0.5cm}\\}
%{$\star$}\\
%\Repeat{12}{\hspace{-5mm}\line(1,0){510}\vspace{0.5cm}\\}
%\Repeat{24}{\line(1,0){510}\vspace{0.5cm}\\}
%\clearpage
%\newpage


% Organization 1
%\subsection*{Esta secci\'on no sera le\'ida ni corregida. Esta secci\'on es para que tu organices tus respuestas (dibujando esquemas, por ej.), si fuera necesario.}
%\clearpage
%\newpage

\end{document}