%LaTeX Curriculum Vitae Template
%
% Copyright (C) 2004-2009 Jason Blevins <jrblevin@sdf.lonestar.org>
% http://jblevins.org/projects/cv-template/
%
% You may use use this document as a template to create your own CV
% and you may redistribute the source code freely. No attribution is
% required in any resulting documents. I do ask that you please leave
% this notice and the above URL in the source code if you choose to
% redistribute this file.

\documentclass[letterpaper]{article}

\usepackage{hyperref}
\hypersetup{
    bookmarks=true,         % show bookmarks bar?
    unicode=false,          % non-Latin characters in Acrobat’s bookmarks
    pdftoolbar=true,        % show Acrobat’s toolbar?
    pdfmenubar=true,        % show Acrobat’s menu?
    pdffitwindow=true,     % window fit to page when opened
    pdfstartview={FitH},    % fits the width of the page to the window
    pdftitle={My title},    % title
    pdfauthor={Author},     % author
    pdfsubject={Subject},   % subject of the document
    pdfcreator={Creator},   % creator of the document
    pdfproducer={Producer}, % producer of the document
    pdfkeywords={keyword1} {key2} {key3}, % list of keywords
    pdfnewwindow=true,      % links in new window
    colorlinks=true,       % false: boxed links; true: colored links
    linkcolor=blue,          % color of internal links (change box color with linkbordercolor)
    citecolor=blue,        % color of links to bibliography
    filecolor=blue,      % color of file links
    urlcolor=blue           % color of external links
}



\usepackage{geometry}
\usepackage{import} % To import email.
\usepackage{marvosym} % face package
\usepackage{xcolor,color}
 \usepackage{fontawesome}

% Comment the following lines to use the default Computer Modern font
% instead of the Palatino font provided by the mathpazo package.
% Remove the 'osf' bit if you don't like the old style figures.
\usepackage[T1]{fontenc}
\usepackage[sc,osf]{mathpazo}

% Set your name here
\def\name{Introducci\'on a las Ciencias Sociales - {\color{red}sigla pendiente}}

% Replace this with a link to your CV if you like, or set it empty
% (as in \def\footerlink{}) to remove the link in the footer:
\def\footerlink{}
% \href{http://www.hectorbahamonde.com}{www.HectorBahamonde.com}

% The following metadata will show up in the PDF properties
\hypersetup{
  colorlinks = true,
  urlcolor = blue,
  pdfauthor = {\name},
  pdfkeywords = {intro to social sciences},
  pdftitle = {\name: Syllabus},
  pdfsubject = {Syllabus},
  pdfpagemode = UseNone
}

\geometry{
  body={6.5in, 8.5in},
  left=1.0in,
  top=1.25in
}

% Customize page headers
\pagestyle{myheadings}
\markright{{\tiny \name}}
\thispagestyle{empty}

% Custom section fonts
\usepackage{sectsty}
\sectionfont{\rmfamily\mdseries\Large}
\subsectionfont{\rmfamily\mdseries\itshape\large}

% Don't indent paragraphs.
\setlength\parindent{0em}

% Make lists without bullets
\renewenvironment{itemize}{
  \begin{list}{}{
    \setlength{\leftmargin}{1.5em}
  }
}{
  \end{list}
}


% email input begin
\newread\fid
\newcommand{\readfile}[1]% #1 = filename
{\bgroup
  \endlinechar=-1
  \openin\fid=#1
  \read\fid to\filetext
  \loop\ifx\empty\filetext\relax% skip over comments
    \read\fid to\filetext
  \repeat
  \closein\fid
  \global\let\filetext=\filetext
\egroup}
\readfile{/Users/hectorbahamonde/RU/Bibliografia_PoliSci/email.txt}
% email input end


%%% bib begin
\usepackage[american]{babel}
\usepackage{csquotes}
%\usepackage[style=chicago-authordate,doi=false,isbn=false,url=false,eprint=false]{biblatex}

\usepackage[authordate,isbn=false,doi=false,url=false,eprint=false]{biblatex-chicago}
\DeclareFieldFormat[article]{title}{\mkbibquote{#1}} % make article titles in quotes
\DeclareFieldFormat[thesis]{title}{\mkbibemph{#1}} % make theses italics

\AtEveryBibitem{\clearfield{month}}
\AtEveryCitekey{\clearfield{month}}

\addbibresource{/Users/hectorbahamonde/RU/Bibliografia_PoliSci/library.bib} 
\addbibresource{/Users/hectorbahamonde/RU/Bibliografia_PoliSci/Bahamonde_BibTex2013.bib} 

% USAGES
%% use \textcite to cite normal
%% \parencite to cite in parentheses
%% \footcite to cite in footnote
%% the default can be modified in autocite=FOO, footnote, for ex. 
%%% bib end




\begin{document}

% Place name at left
%{\huge \name}

% Alternatively, print name centered and bold:
\centerline{\huge \bf \name}

\vspace{0.25in}

\begin{minipage}{0.45\linewidth}
 Universidad de O$'$Higgins \\
  Instituto de Ciencias Sociales \\
  Rancagua, Chile\\
  \\
  \\

\end{minipage}
\hspace{4cm}\begin{minipage}{0.45\linewidth}
  \begin{tabular}{ll}
{\bf \'Ultima actualizaci\'on}: \today. \\
 {\bf Descarga la \'ultima versi\'on} \href{https://github.com/hbahamonde/Intro_Ciencias_Sociales/raw/master/Bahamonde_Intro_Ciencias_Sociales.pdf}{aqu\'i}.%\\
   %{\bf {\color{red}{\scriptsize Not intended as a definitive version}}} %\\
    \\
    \\
    \\
    \\
    \\
  \end{tabular}
\end{minipage}

\subsection*{Aspectos Log\'isticos}


\vspace{1mm}
{\bf Profesor}: H\'ector Bahamonde, PhD.\\
%\texttt{e:}\href{mailto:hbahamonde@tulane.edu}{\texttt{hbahamonde@tulane.edu}}\\
\texttt{e:}\href{mailto:hector.bahamonde@uoh.cl}{\texttt{hector.bahamonde@uoh.cl}}\\
\texttt{w:}\href{http://www.hectorbahamonde.com}{\texttt{www.hectorbahamonde.com}}\\
{\bf Office Hours}: Toma una hora \href{https://calendly.com/bahamonde/officehours}{\texttt{aqu\'i}}.\\
{\bf Acceso a materiales del curso}: {\color{red}uCampus, pendiente}.

\vspace{5mm}
{\bf \underline{Administraci\'on P\'ublica}}
\\
\\
{\bf Hora de c\'atedra}: {\input{/Users/hectorbahamonde/RU/Teaching/Intro_Ciencias_Sociales/time_class_1_AP.txt}\unskip}; {\input{/Users/hectorbahamonde/RU/Teaching/Intro_Ciencias_Sociales/time_class_2_AP.txt}\unskip}.\\
{\bf Lugar de c\'atedra}: {\color{red}pendiente}.\\

\vspace{2mm}
{\bf Ayudante de c\'atedra (TA)}: {\input{/Users/hectorbahamonde/RU/Teaching/Intro_Ciencias_Sociales/ta_name_1.txt}\unskip}.\\
\texttt{e:}\href{mailto:paz.castro@pregrado.uoh.cl}{\texttt{paz.castro@pregrado.uoh.cl}}\\
{\bf TA Bio}: {\input{/Users/hectorbahamonde/RU/Teaching/Intro_Ciencias_Sociales/ta_name_1.txt}\unskip} es una excelente estudiante de Administraci\'on P\'ublica UOH.\\
{\bf Hora de ayudant\'ia}: {\input{/Users/hectorbahamonde/RU/Teaching/Intro_Ciencias_Sociales/recitation_day_AP.txt}\unskip}: {\input{/Users/hectorbahamonde/RU/Teaching/Intro_Ciencias_Sociales/recitation_time_AP.txt}\unskip}.\\
{\bf Lugar de ayudant\'ia}: {\color{red}sala pendiente}.



\vspace{5mm}
{\bf \underline{Ingenier\'ia Comercial}}
\\
\\
{\bf Hora de c\'atedra}: {\input{/Users/hectorbahamonde/RU/Teaching/Intro_Ciencias_Sociales/time_class_1_IC.txt}\unskip}; {\input{/Users/hectorbahamonde/RU/Teaching/Intro_Ciencias_Sociales/time_class_2_IC.txt}\unskip}.\\
{\bf Lugar de c\'atedra}: {\color{red}pendiente}.\\

\vspace{2mm}
{\bf Ayudante de c\'atedra (TA)}: {\input{/Users/hectorbahamonde/RU/Teaching/Intro_Ciencias_Sociales/ta_name_2.txt}\unskip}.\\
\texttt{e:}\href{mailto:diego.astudillo@pregrado.uoh.cl}{\texttt{diego.astudillo@pregrado.uoh.cl}}\\
{\bf TA Bio}: {\input{/Users/hectorbahamonde/RU/Teaching/Intro_Ciencias_Sociales/ta_name_2.txt}\unskip} es un excelente estudiante de Administraci\'on P\'ublica UOH.\\
{\bf Hora de ayudant\'ia}: {\input{/Users/hectorbahamonde/RU/Teaching/Intro_Ciencias_Sociales/recitation_day_IC.txt}\unskip}: {\input{/Users/hectorbahamonde/RU/Teaching/Intro_Ciencias_Sociales/recitation_time_IC.txt}\unskip}.\\
{\bf Lugar de ayudant\'ia}: {\color{red}sala pendiente}.



\subsection*{Overview y Objetivos}

Este {\color{blue}curso de pregrado} ofrece una introducci\'on a los conceptos y tem\'aticas---y sobre todo, al lenguaje---que nosotros, los cientistas, sociales usamos a diario. El curso est\'a dise\~nado tanto para estudiantes de Administraci\'on P\'ublica como Ingenieros Comerciales (ambas menciones). 
\\
\\
El curso busca que los estudiantes comprendan los enfoques, conceptos y procesos fundamentales que da cuenta no s\'olo la sociolog\'ia, si no que tambi\'en la econom\'ia, la ciencia pol\'itica, y la historia, para explicar el contexto en el cual se desarrollan los asuntos p\'ublicos. Interesa que el estudiante sea capaz de reconocer no s\'olo c\'omo las instituciones culturales, si no que tambi\'en las econ\'omicas y sociales condicionan el comportamiento humano.


\begin{itemize}
    \item[\Pointinghand] El curso se estructura en torno a clases expositivas hechas por el profesor. Aunque habr\'a un \'unico programa, {\color{blue}tendremos sesiones de c\'atedra por separado}. Las ayudant\'ias tambi\'en se har\'an por separado.
\end{itemize}


Se espera que los estudiantes hagan sus respectivas lecturas \emph{antes} de cada clase para poder participar en el debate cr\'itico que haremos en cada una de ellas. En particular, las lecturas ser\'an en su gran mayor\'ia libros catalogados como ``populares'' o ``masivos''. Lo interesante es que estos libros han sido escritos por acad\'emicos y acad\'emicas de primer nivel mundial, pero que primero han escritos libros altamente t\'ecnicos. En otras palabras, los libros que leeremos corresponden a las versiones ``populares'' de estas versiones m\'as t\'ecnicas. Nota que los autores de ambas versiones (i.e. popular y t\'ecnica) son los mismos.
\\
\\
Las clases---y en consecuencia, el semestre---estar\'an divididas en las tres \'areas m\'as grandes de las ciencias sociales, esto es, la econom\'ia, la pol\'itica, y la sociedad. Dentro de cada secci\'on no s\'olo leeremos estos libros, si no que tambi\'en analizaremos otros materiales, como pel\'iculas y documentales. Debido a que este curso est\'a principalmente avocado al an\'alisis critico de lecturas, cada una de estas \'areas ser\'a evaluada por separado. Al final del semestre, sin embargo, tendremos un ensayo final que intentar\'a medir cu\'an bien t\'u puedes integrar los distintos conceptos revisados en clases (y presentes en las lecturas) entre estas tres \'areas. 
\\
\\
Este es un curso de primer semestre que tambi\'en intenta ense\~narte herramientas que te ser\'an \'utiles para el resto de tu permanencia en la Universidad. {\bf En mi curso aprender\'as a leer de nuevo}. No s\'olo tendr\'as que leer a vol\'umenes que jam\'as hayas visto antes, si no que tambi\'en aprender\'as a leer \emph{cr\'iticamente}. {\bf En mi curso aprender\'as a escribir nuevo}. Esto debido a que tambi\'en tendr\'as que desarrollar habilidades que quiz\'as antes no ten\'ias, como es escribir ensayos. Ya conversaremos c\'omo un \emph{ensayo} es diferente a un \emph{resumen}. Finalmente, tambi\'en tendr\'as que aprender a debatir argumentos l\'ogicos en clases. 
\\
\\
Honestamente, espero que este curso cautive tu atenci\'on, y simiente tu curiosidad intelectual, sobre todo, mostr\'andote que nuestro objeto de estudio (la sociedad) es apasionante. 
\\
\\
\emph{Bienvenid$@$s!}

\subsection*{Objetivos Generales del Curso}
 
Al finalizar este curso, tu podr\'as:

\begin{itemize}
	\item[$\bullet$] Manejar los principales debates y problemas fundamentales de nuestra sociedad. 
	\item[$\bullet$] Integrar distintos enfoques, a trav\'es de distintas disciplinas.
	\item[$\bullet$] Analizar cr\'iticamente textos y otros materiales (como pel\'iculas y documentales).
	\item[$\bullet$] Elaborar argumentos l\'ogicos.
	\item[$\bullet$] Consumir literatura m\'as especializada en tus sucesivos semestres.
  \item[$\bullet$] Construir ensayos en base a tus propias hip\'otesis.
\end{itemize}


\subsection*{Etiquette}
 

\begin{itemize}
	\item[$\bullet$] No llegues tarde. La sala de clases se cierra despu\'es de los primeros 15 minutos. Esta regla es importante, y no tendr\'a excepciones.
	\item[$\bullet$] No comas en clases. Bebestibles, tales como caf\'e y t\'e est\'an OK.
	\item[$\bullet$] No se pueden ocupar \emph{laptops}, celulares, \emph{tablets}, ni otros aparatos digitales. No habr\'an excepciones. Los celulares deber\'an estar apagados, no en silencio. Aquellos estudiantes que no respeten esta regla, ser\'an invitados a salir de la sala. No puedes sacar fotos a las diapositivas.
	\item[$\bullet$] La asistencia es obligatoria (y parte de tu nota de participaci\'on). Si faltaste a una clase, cons\'iguete las notas con un compa\~nero/a. Yo no ofrezco clases particulares de mi clase. Sin embargo, si tienes preguntas especificas, \href{https://calendly.com/bahamonde/officehours}{toma una hora} conmigo. 
\end{itemize}




\subsection*{Evaluaciones}

\begin{enumerate}

	% Participation
	\item {\bf Lecturas, Participaci\'on, y Asistencia}: {\input{/Users/hectorbahamonde/RU/Teaching/Intro_Ciencias_Sociales/percentage_participation.txt}\unskip}\%.
	\\
	\\
	El/la ayudante y yo asumiremos durante todo el semestre que has le\'ido. Nosotros empleamos un m\'etodo de clases interactivo, pero este m\'etodo necesita de tu participaci\'on activa en clases. \emph{Asistencia no s\'olo significa que vayas a clase; tambi\'en debes participar}. Esto significa que aunque hayas ido a todas las clases, es \emph{imposible} que tengas la nota m\'axima en asistencia si es que no has participado en clases y ayudant\'ia. Es por esto que la nota es de asistencia \emph{y} participaci\'on.
\\
\\	
	Para asegurarnos de que est\'es haciendo las lecturas, habr\'an una serie de \emph{pop quizzes} (``pruebas sorpresa'') tanto en c\'atedra como ayudant\'ia. Estos controles ser\'an cortos (3-5 minutos), y apuntan a medir si leyeron; o sabes, o no. Estas pruebas se aplicar\'an completamente al azar, en cualquier momento de la clase, y sin previo aviso. Si faltaste, tendr\'as un 1 en ese control. En general, las preguntas ser\'an del tipo: ``\emph{cu\'al es el pa\'is que el autor usa para ilustrar su ejemplo?}.'' Aqu\'i la respuesta contiene s\'olo \emph{una} palabra.
	%There will be a number of pop quizzes during the semester, both in lecture \emph{and} recitation. Quizzes will be short (3-5 minutes), completed at any point of the class, and designed to make sure everyone is keeping up with the readings and lecture. There will be no make-up quizzes. If you are absent (or late) from class that day, you will get a $1$ on that quiz. 
	%\\
	%\\
	% 3 credits
	%Students are expected to put in 90 hours of work during the semester for a 3-credit class. That represents 5 hours per week, in a semester of 18 weeks. These are \emph{Universidad de O\'\unskip Higgins}'s guidelines. Since you will be spending 1.5 hours in the classroom, this means you should be working about 3.5 hours per week for this course {\bf outside} of the classroom. If you find that you are spending more than that, please see me in my office hours to discuss strategies to read more efficiently. 
	\\
	\\
	% 6 credits
	%Se espera que los estudiantes trabajen 180 horas durante el semestre para un curso como este, de 6 cr\'editos. Eso significa que semanalmente, tendr\'as que trabajar 10 horas, en un semestre de 18 semanas. {\color{red}pendiente: revisar numero de semanas.}
	%Students are expected to put in 180 hours of work during the semester for a 6-credit class. That represents 10 hours per week, in a semester of 18 weeks. These are \emph{Universidad de O$'$Higgins}'s guidelines. Since you will be spending 3 hours in the classroom, this means you should be working about 7 hours per week for this course {\bf outside} of the classroom. Since recitation lasts for 1.5 hours, that means that you should be {\bf reading} 5.5 hours per week. If you find that you are spending more than that, please see me in my office hours to discuss strategies to read more efficiently. 

	% pruebas tematicas
	\item {\bf Tres pruebas t\'ematicas}: {\input{/Users/hectorbahamonde/RU/Teaching/Intro_Ciencias_Sociales/percentage_prueba_seccion.txt}\unskip}\% cada una.

		\begin{itemize}
			\item[$\bullet$] Econom\'ia: {\input{/Users/hectorbahamonde/RU/Teaching/Intro_Ciencias_Sociales/date_seccion_economia.txt}\unskip}.
			\item[$\bullet$] Pol\'itica: {\input{/Users/hectorbahamonde/RU/Teaching/Intro_Ciencias_Sociales/date_seccion_politica.txt}\unskip}.
			\item[$\bullet$] Sociedad: {\input{/Users/hectorbahamonde/RU/Teaching/Intro_Ciencias_Sociales/date_seccion_sociedad.txt}\unskip}.
		\end{itemize}

Tienes que dar estas pruebas en la fecha establecida. No habr\'an pruebas recuperativas. La \'unica excusa valida ser\'a de car\'acter m\'edico. Las preguntas saldr\'an de las lecturas, las clases, y las ayudant\'ias.


% Reaction Papers
	\item {\bf Seis \emph{Reaction Papers}}, fecha libre: {\input{/Users/hectorbahamonde/RU/Teaching/Intro_Ciencias_Sociales/percentage_reaction_paper.txt}\unskip}\% cada uno.
	\\ 
	\\
	En este ejercicio, escribir\'as seis ``\emph{papers} de reacci\'on'' de dos p\'aginas como m\'aximo---{\emph{papers}} m\'as largos, no ser\'an corregidos mas all\'a del limite establecido. Ya aprenderemos lo que es un \emph{paper} de reacci\'on. Esencialmente, el/la ayudante queremos saber \emph{qu\'e piensas t\'u de la lectura}. {\bf No queremos res\'umenes}: los res\'umenes ser\'an calificados con un 1. Espec\'ificamente, queremos saber: 

	\begin{itemize}
		\item[$\bullet$] Qu\'e te llam\'o la atenci\'on? Alg\'un hallazgo/pa\'is/situaci\'on/hecho que t\'u conozcas, pero que no calce con la teor\'ia del/la autor/a?
		\item[$\bullet$] Qu\'e hubieras hecho diferente, si t\'u hubieras sido el/la autor/a?
		\item[$\bullet$] Se habr\'a olvidado de algo el/la autor/a?
	\end{itemize}

	Si te fijas bien, para esta actividad no hay fecha. T\'u manejas tus tiempos. Lo que queremos es que escojas los libros que m\'as te llamen la atenci\'on, y el d\'ia donde discutamos ese libro, tu entregues un \emph{reaction paper} al comienzo de la clase. 
\\
\\
	Presta atenci\'on a las siguientes instrucciones:

	\begin{itemize}
		\item[$\bullet$] Debes presentar dos \emph{reaction papers} por secci\'on. Esto es, dos \emph{reaction papers} para la secci\'on de econom\'ia, dos \emph{reaction papers} para la secci\'on de pol\'itica, y dos \emph{reaction papers} para la secci\'on de sociedad.
		\item[$\bullet$] Los \emph{papers} se entregan {\bf impresos} y {\bf dentro de los primeros 15 minutos de clase}. Desde el minuto 16 en adelante no se recibir\'an \emph{papers}. \emph{Papers} entregados v\'ia \emph{email} no ser\'an aceptados bajo ninguna circunstancia.
		\item[$\bullet$] Si lees el libro X, el \emph{reaction paper} del libro X se entrega el d\'ia en el cual esta designado la lectura del libro X; no en otro d\'ia.
		\item[$\bullet$] No habr\'an \emph{reaction papers} acerca de documentales.
		\item[$\bullet$] Organ\'izate bien: t\'u decides cu\'ando entregar un \emph{reaction paper}. Sin embargo, cada \emph{reaction paper} que te falte ser\'a calificado con un 1.
	\end{itemize}


%

	\item {\bf Un ensayo final}, {\input{/Users/hectorbahamonde/RU/Teaching/Intro_Ciencias_Sociales/date_ensayo_final.txt}\unskip}: {\input{/Users/hectorbahamonde/RU/Teaching/Intro_Ciencias_Sociales/percentage_ensayo_final.txt}\unskip}\%.


En este curso, la actividad final es un ensayo ``take-home'' de 10-15 p\'aginas. En base a un temario que se entregar\'a el {\input{/Users/hectorbahamonde/RU/Teaching/Intro_Ciencias_Sociales/date_temario.txt}\unskip}, tendr\'as que escribir un ensayo cr\'itico de las lecturas, seguramente integrando varias secciones. {\color{red}En su debido momento, conversaremos los pormenores del ensayo (formato, bibligraf\'ia externa, citas, etc.)}. 



	\begin{itemize}
		\item[\Pointinghand] Nosotros tomamos la escritura de ensayos muy seriamente. En consecuencia, nosotros recomendamos que escriban sus \emph{papers} con la debida anticipaci\'on, editen varios borradores, pregunten al TA y a mi, env\'ien preguntas y borradores al profesor/ayudante. El correcto uso de la gram\'atica, dicci\'on, y estilo impactar\'an (para bien o para mal) la nota de este producto. Consulten William Strunk, Jr., and E. B. White, \href{http://www.jlakes.org/ch/web/The-elements-of-style.pdf}{\emph{The Elements of Style}}, para revisar algunos \emph{tips} de c\'omo escribir en la universidad. Joseph M. Williams and Gregory G. Colomb, \href{http://sir.spbu.ru/en/programs/master/master_program_in_international_relations/digital_library/Book%20Research%20seminar%20by%20Booth.pdf}{\emph{The Craft of Argument}} (New York: Longman, 2003), proveen una excelente revisi\'on de escritura persuasiva.\phantom{\textcite{Strunk1999,Booth2003}}
	\end{itemize}


	{\bf El paper se entrega \underline{impreso} dentro de los primeros 15 minutos de clase}, el {\bf {\input{/Users/hectorbahamonde/RU/Teaching/Intro_Ciencias_Sociales/date_ensayo_final.txt}\unskip}}. Se puede entregar antes, pero desde el minuto 16 de ese d\'ia, {\bf no se recibir\'an ensayos}. Los \emph{papers} que se entreguen tarde o v\'ia \emph{email} tendr\'an un 1 (sin opci\'on a recorrecci\'on). 
	\\
	\\
	Les recomiendo verme en \href{https://calendly.com/bahamonde/officehours}{mis office hours} \emph{antes} del plazo de entrega. Si quieres, \href{mailto:\filetext}{env\'iame un email} con tu borrador, y yo te devolver\'e comentarios. V\'elo como una pre-correcci\'on. Esto es voluntario. Tambi\'en puedes contactar al/la TA. 

	\begin{itemize}
		\item[\Pointinghand] Los ayudantes y yo hemos escrito \href{https://github.com/hbahamonde/Intro_Ciencias_Sociales/raw/master/Ensayo_Final/Bahamonde_Ensayo_Checklist.pdf}{un documento} que contiene un \emph{checklist} del ensayo final.
	\end{itemize}

\end{enumerate}



\subsection*{Ayudant\'ia}

Cada semana te reunir\'as con tu ayudante (``TA''). Ah\'i tendr\'as otra oportunidad para conversar acerca de los textos, y seguir profundizando otras tem\'aticas pendientes. Tambi\'en sera una oportunidad para revisar pel\'iculas y documentales relevantes. En esta oportunidad, tambi\'en se revisar\'an aspectos m\'as formales de las humanidades y las ciencias sociales. Por ejemplo, aprender\'as a citar, aprender\'as c\'omo estructurar un ensayo, entre otros aspectos.
\\
\\
{\color{red}pendiente: fijar las p\'aginas para cada lectura}


\subsection*{Calendario}

\begin{enumerate}

\item {\bf Introducci\'on}

      \begin{itemize}
        \item[$\bullet$] Jared Diamond. 2016. \href{https://github.com/hbahamonde/Intro_Ciencias_Sociales/raw/master/Readings/Armas_germenes_acero_Diamond.pdf}{\emph{Armas, G\'ermenes y Acero: Breve Historia de la Humanidad en los \'Ultimos Trece Mil A\~nos}}. Debolsillo.\phantom{\textcite{Diamond:2016aa}}
        \item[$\bullet$] Yuval Harari. 2014. \href{https://github.com/hbahamonde/Intro_Ciencias_Sociales/raw/master/Readings/De_Animales_Dioses_Harari.pdf}{\emph{Sapiens. De Animales a Dioses: Una Breve Historia de la Humanidad}}. Debate.\phantom{\textcite{Harari:2014aa}}
      \end{itemize}





\item {\bf Econom\'ia}

      \begin{itemize}
        \item[$\bullet$] David Harvey. 2007. \href{https://github.com/hbahamonde/Intro_Ciencias_Sociales/raw/master/Readings/Breve_histora_Neoliberalismo_Harvey.pdf}{\emph{Breve Historia del Neoliberalismo}}. Ediciones Akal.\phantom{\textcite{Harvey:2007aa}}
        \item[$\bullet$] Thomas Piketty. 2015. \href{https://github.com/hbahamonde/Intro_Ciencias_Sociales/raw/master/Readings/Piketty_El_capital.pdf}{\emph{El Capital en el Siglo XXI}}. RBA Libros.\phantom{\textcite{Piketty:2015aa}}
        \item[$\bullet$] Naomi Klein. 2012. \href{https://github.com/hbahamonde/Intro_Ciencias_Sociales/raw/master/Readings/La_doctrina_del_shock_Klein.pdf}{\emph{La Doctrina del Shock: El Auge del Capitalismo del Desastre}}.Booket.\phantom{\textcite{Klein:2012aa}}
      \end{itemize}


\item {\bf Pol\'itica}

      \begin{itemize}
        \item[$\bullet$] Daron Acemoglu y James Robinson. 2014. \href{https://github.com/hbahamonde/Intro_Ciencias_Sociales/raw/master/Readings/Why_Nations_Fail.pdf}{\emph{Por qu\'e Fracasan los Pa\'ises: Los Or\'igenes del Poder, la Prosperidad y la Pobreza}}. Booket.\phantom{\textcite{Acemoglu:2014aa}}
        \item[$\bullet$] Steven Levitsky y Daniel Ziblatt. 2018. \emph{C\'omo Mueren las Democracias}. Ariel.{\color{red}pending}\phantom{\textcite{Levitsky:2018ab}}
      \end{itemize}

\item {\bf Sociedad}

      \begin{itemize}
        \item[$\bullet$] Jared Diamond. 2012. \href{https://github.com/hbahamonde/Intro_Ciencias_Sociales/raw/master/Readings/Diamond_Colapso_sociedades_perduran_desaparecen.epub}{\emph{Colapso: Por qu\'e unas Sociedades Perduran y otras Desaparecen}}. Debate Editorial.\footnote{Para este texto, debes tener un {\color{blue}\texttt{ePub}} reader. Consulta con los ayudantes d\'onde/c\'omo obtener uno.}\phantom{\textcite{Diamond:2012aa}}
        \item[$\bullet$] Malcolm Gladwell. 2011. \emph{El Punto Clave: C\'omo los Peque\~nos Cambios Pueden Provocar Grandes Efectos}. Punto de Lectura.{\color{red}pending}\phantom{\textcite{Gladwell:2011aa}}
      \end{itemize}


\end{enumerate}








 








\newpage
\pagenumbering{roman}
\setcounter{page}{1}
\printbibliography



\end{document}

